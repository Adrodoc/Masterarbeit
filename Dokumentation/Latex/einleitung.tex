\chapter{Einleitung}

\section{Motivation}

Locks sind einer der grundlegendsten und verbreitetsten Mechanismen für die Synchronisierung von Speicherzugriffen in parallelen Programmen
und es gibt eine Vielzahl von Algorithmen,
die auf Locks basieren.
Es ist daher nicht verwunderlich,
dass immer wieder neue Lock-Algorithmen entwickelt werden,
die weitere Verbesserungen in Geschwindigkeit, Speicherverbrauch oder anderen Bereichen erzielen.
Die meisten dieser Algorithmen werden für Systeme mit gemeinsamem Speicher entwickelt.
Locks können aber auch für die Programmierung in verteilten Systemen,
z.~B. im Bereich \gls{hpc},
nützlich sein \cite{RMA-RW}.

Da in verteilten Systemen die Zugriffe auf entfernten Speicher deutlich langsamer sind
als auf lokalen Speicher,
müssen entfernte Zugriffe vermieden werden,
um eine gute Performance zu erzielen.
Locks, die nicht mit diesem Ziel entworfen wurden,
eignen sich daher weniger gut für den Einsatz in verteilten Systemen.

Diese Anforderung,
Zugriffe auf entfernten Speicher zu vermeiden,
gibt es aber nicht nur bei verteilten Systemen,
sondern auch bei Systemen mit gemeinsamem Speicher:
Manche Prozessoren nutzen Architekturen mit \glsfirst{numa}.
Hierbei sind Zugriffe auf den Speicher anderer Prozessoren langsamer
als Zugriffe auf den eigenen Speicher \cite{NUMA}
und müssen entsprechend vermieden werden,
um eine gute Performance zu erzielen.
Algorithmen,
die für \gls{numa}-Architekturen entworfen wurden,
könnten sich daher auch für verteilte Systeme eignen.

Im Rahmen dieser Arbeit soll daher untersucht werden,
ob durch die Portierung von Locks für \gls{numa}-Architekturen auf verteilten Speicher
eine bessere Performance erreicht werden kann
als durch Nutzung bestehender Implementierungen für verteilte Systeme.

\section{Übersicht}

In \autoref{ch:hinfuehrung} werden die notwendigen Grundlagen zu Locks,
den Unterschieden zwischen gemeinsamem und verteiltem Speicher
und \gls{numa} etabliert.
Darauf aufbauend wird das Ziel der Arbeit hergeleitet.

\autoref{ch:dist_locks} zeigt dann,
welche Locks bereits auf verteiltem Speicher zur Verfügung stehen
und mit welchen Optimierungen diese verbessert werden können.
Neben einem Lock basierend auf \texttt{MPI\_Win\_lock}
werden dabei drei weitere Locks betrachtet:
\texttt{dash::Mutex} aus \cite{DART-MPI},
sowie D-MCS und RMA-MCS aus \cite{RMA-RW}.

Da es bisher kaum Locks für verteilte Systeme gibt,
gibt es auch keine Benchmarksuites,
mit denen neue Implementierungen detailliert mit Bestehenden verglichen werden könnten.
Daher werden in \autoref{ch:benchmarks} die relevanten Kriterien
für einen Vergleich zusammengefasst
und eine neue Benchmarksuite entwickelt.
Mit dieser können
die Geschwindigkeit,
die praktische Fairness
und andere algorithmusspezifische Kennzahlen
von beliebigen Lock-Implementierungen auf verteiltem Speicher
in verschiedenen Szenarien gemessen werden.

Anschließend werden in \autoref{ch:portierung}
die sieben Lock-Algorithmen für \gls{numa} aus \autoref{tab:numa_locks} analysiert,
um zu entscheiden,
welche der Algorithmen sich für eine Portierung auf verteilten Speicher eignen.
Die geeigneten Algorithmen werden dann portiert,
anhand der Erkenntnisse aus \autoref{ch:dist_locks} optimiert
und mit der Benchmarksuite aus \autoref{ch:benchmarks} evaluiert.

\begin{table}[h]
    \centering
    \begin{tabular}{|c|c|c|c|}
        \hline
        Algorithmus & Jahr & Quelle             & Analyse                   \\
        \hline
        RH-Lock     & 2002 & \cite{RH-Lock}     & \autoref{sec:rh-lock}     \\
        \hline
        HCLH-Lock   & 2006 & \cite{HCLH-Lock}   & \autoref{sec:hclh-lock}   \\
        \hline
        Cohort-Lock & 2012 & \cite{Cohort-Lock} & \autoref{sec:cohort-lock} \\
        \hline
        HMCS-Lock   & 2015 & \cite{HMCS-Lock}   & \autoref{sec:cohort-lock} \\
        \hline
        AHMCS-Lock  & 2016 & \cite{AHMCS-Lock}  & \autoref{sec:ahmcs-lock}  \\
        \hline
        CST-Lock    & 2017 & \cite{CST-Lock}    & \autoref{sec:cst-lock}    \\
        \hline
        SHFL-Lock   & 2019 & \cite{Shfl-Lock}   & \autoref{sec:shfl-lock}   \\
        \hline
    \end{tabular}
    \caption{Untersuchte NUMA-Locks}
    \label{tab:numa_locks}
\end{table}

Schließlich werden in \autoref{ch:fazit} die Ergebnisse zusammengefasst.
