\vspace*{2cm}

\begin{center}
    \textbf{Abstract}
\end{center}

\vspace*{1cm}

\noindent Locks sind einer der wichtigsten Grundbausteine für die parallele Programmierung.
Trotzdem gibt es kaum Lock-Implementierungen für verteilten Speicher,
dafür aber eine Vielzahl an Lock-Algorithmen für gemeinsamen Speicher,
die \gls{numa} berücksichtigen.
Genau wie bei verteiltem Speicher
sind bei \gls{numa} Zugriffe auf entfernten Speicher deutlich langsamer
und müssen vermieden werden.
Daher wird in dieser Arbeit eine Auswahl solcher Algorithmen
auf verteilten Speicher portiert
und dort optimiert.
Für die Evaluation wird eine Benchmarksuite entwickelt,
mit der die Geschwindigkeit,
die praktische Fairness
und andere algorithmusspezifische Kennzahlen
von beliebigen Lock-Implementierungen auf verteiltem Speicher
in verschiedenen Szenarien gemessen werden kann.
Die Evaluation zeigt,
dass durch Portierung und Optimierung eines Cohort-Locks
mit globalem MCS-Lock und lokalem Hemlock
eine Implementierung entwickelt wurde,
die bis zu vier Mal so schnell ist
wie der bisher beste Lock auf verteiltem Speicher,
der RMA-MCS-Lock.
Dabei hat die neue Implementierung einen geringeren Speicherverbrauch
und bietet vergleichbare Fairness.
